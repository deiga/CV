%%%%%%%%%%%%%%%%%%%%%%%%%%%%%%%%%%%%%%%%%%%%%%%%%%%%%%%%%%%%%%%%%%%%%%%%
%%%%%%%%%%%%%%%%%%%%%% Simple LaTeX CV Template %%%%%%%%%%%%%%%%%%%%%%%%
%%%%%%%%%%%%%%%%%%%%%%%%%%%%%%%%%%%%%%%%%%%%%%%%%%%%%%%%%%%%%%%%%%%%%%%%

%%%%%%%%%%%%%%%%%%%%%%%%%%%%%%%%%%%%%%%%%%%%%%%%%%%%%%%%%%%%%%%%%%%%%%%%
%% NOTE: If you find that it says                                     %%
%%                                                                    %%
%%                           1 of ??                                  %%
%%                                                                    %%
%% at the bottom of your first page, this means that the AUX file     %%
%% was not available when you ran LaTeX on this source. Simply RERUN  %%
%% LaTeX to get the ``??'' replaced with the number of the last page  %%
%% of the document. The AUX file will be generated on the first run   %%
%% of LaTeX and used on the second run to fill in all of the          %%
%% references.                                                        %%
%%%%%%%%%%%%%%%%%%%%%%%%%%%%%%%%%%%%%%%%%%%%%%%%%%%%%%%%%%%%%%%%%%%%%%%%

%%%%%%%%%%%%%%%%%%%%%%%%%%%% Document Setup %%%%%%%%%%%%%%%%%%%%%%%%%%%%

% Don't like 10pt? Try 11pt or 12pt
\documentclass[10pt,finnish,a4paper]{article}
\usepackage{babel}
\usepackage[utf8]{inputenc}


% This is a helpful package that puts math inside length specifications
\usepackage{calc}

% Simpler bibsection for CV sections
% (thanks to natbib for inspiration)
\makeatletter
\newlength{\bibhang}
\setlength{\bibhang}{1em}
\newlength{\bibsep}
 {\@listi \global\bibsep\itemsep \global\advance\bibsep by\parsep}
\newenvironment{bibsection}
    {\minipage[t]{\linewidth}\list{}{%
        \setlength{\leftmargin}{\bibhang}%
        \setlength{\itemindent}{-\leftmargin}%
        \setlength{\itemsep}{\bibsep}%
        \setlength{\parsep}{\z@}%
        }}
    {\endlist\endminipage}
\makeatother

% Layout: Puts the section titles on left side of page
\reversemarginpar

%
%         PAPER SIZE, PAGE NUMBER, AND DOCUMENT LAYOUT NOTES:
%
% The next \usepackage line changes the layout for CV style section
% headings as marginal notes. It also sets up the paper size as either
% letter or A4. By default, letter was used. If A4 paper is desired,
% comment out the letterpaper lines and uncomment the a4paper lines.
%
% As you can see, the margin widths and section title widths can be
% easily adjusted.
%
% ALSO: Notice that the includefoot option can be commented OUT in order
% to put the PAGE NUMBER *IN* the bottom margin. This will make the
% effective text area larger.
%
% IF YOU WISH TO REMOVE THE ``of LASTPAGE'' next to each page number,
% see the note about the +LP and -LP lines below. Comment out the +LP
% and uncomment the -LP.
%
% IF YOU WISH TO REMOVE PAGE NUMBERS, be sure that the includefoot line
% is uncommented and ALSO uncomment the \pagestyle{empty} a few lines
% below.
%

%% Use these lines for letter-sized paper
\usepackage[paper=letterpaper,
            %includefoot, % Uncomment to put page number above margin
            marginparwidth=1.2in,     % Length of section titles
            marginparsep=.05in,       % Space between titles and text
            margin=1in,               % 1 inch margins
            includemp]{geometry}

%% Use these lines for A4-sized paper
%\usepackage[paper=a4paper,
%            %includefoot, % Uncomment to put page number above margin
%            marginparwidth=30.5mm,    % Length of section titles
%            marginparsep=1.5mm,       % Space between titles and text
%            margin=25mm,              % 25mm margins
%            includemp]{geometry}

%% More layout: Get rid of indenting throughout entire document
\setlength{\parindent}{0in}

%% This gives us fun enumeration environments. compactitem will be nice.
\usepackage{paralist}

%% Reference the last page in the page number
%
% NOTE: comment the +LP line and uncomment the -LP line to have page
%       numbers without the ``of ##'' last page reference)
%
% NOTE: uncomment the \pagestyle{empty} line to get rid of all page
%       numbers (make sure includefoot is commented out above)
%
\usepackage{fancyhdr,lastpage}
\pagestyle{fancy}
%\pagestyle{empty}      % Uncomment this to get rid of page numbers
\fancyhf{}\renewcommand{\headrulewidth}{0pt}
\fancyfootoffset{\marginparsep+\marginparwidth}
\newlength{\footpageshift}
\setlength{\footpageshift}
          {0.5\textwidth+0.5\marginparsep+0.5\marginparwidth-2in}
\lfoot{\hspace{\footpageshift}%
       \parbox{4in}{\, \hfill %
                    \arabic{page} of \protect\pageref*{LastPage} % +LP
%                    \arabic{page}                               % -LP
                    \hfill \,}}

% Finally, give us PDF bookmarks
\usepackage{color,hyperref}
\definecolor{darkblue}{rgb}{0.0,0.0,0.3}
\hypersetup{colorlinks,breaklinks,
            linkcolor=darkblue,urlcolor=darkblue,
            anchorcolor=darkblue,citecolor=darkblue}

%%%%%%%%%%%%%%%%%%%%%%%% End Document Setup %%%%%%%%%%%%%%%%%%%%%%%%%%%%


%%%%%%%%%%%%%%%%%%%%%%%%%%% Helper Commands %%%%%%%%%%%%%%%%%%%%%%%%%%%%

% The title (name) with a horizontal rule under it
%
% Usage: \makeheading{name}
%
% Place at top of document. It should be the first thing.
\newcommand{\makeheading}[1]%
        {\hspace*{-\marginparsep minus \marginparwidth}%
         \begin{minipage}[t]{\textwidth+\marginparwidth+\marginparsep}%
                {\large \bfseries #1}\\[-0.15\baselineskip]%
                 \rule{\columnwidth}{1pt}%
         \end{minipage}}

% The section headings
%
% Usage: \section{section name}
%
% Follow this section IMMEDIATELY with the first line of the section
% text. Do not put whitespace in between. That is, do this:
%
%       \section{My Information}
%       Here is my information.
%
% and NOT this:
%
%       \section{My Information}
%
%       Here is my information.
%
% Otherwise the top of the section header will not line up with the top
% of the section. Of course, using a single comment character (%) on
% empty lines allows for the function of the first example with the
% readability of the second example.
\renewcommand{\section}[2]%
        {\pagebreak[2]\vspace{1.3\baselineskip}%
         \phantomsection\addcontentsline{toc}{section}{#1}%
         \hspace{0in}%
         \marginpar{
         \raggedright \scshape #1}#2}

% An itemize-style list with lots of space between items
\newenvironment{outerlist}[1][\enskip\textbullet]%
        {\begin{itemize}[#1]}{\end{itemize}%
         \vspace{-.6\baselineskip}}

% An environment IDENTICAL to outerlist that has better pre-list spacing
% when used as the first thing in a \section
\newenvironment{lonelist}[1][\enskip\textbullet]%
        {\vspace{-\baselineskip}\begin{list}{#1}{%
        \setlength{\partopsep}{0pt}%
        \setlength{\topsep}{0pt}}}
        {\end{list}\vspace{-.6\baselineskip}}

% An itemize-style list with little space between items
\newenvironment{innerlist}[1][\enskip\textbullet]%
        {\begin{compactitem}[#1]}{\end{compactitem}}

% To add some paragraph space between lines.
% This also tells LaTeX to preferably break a page on one of these gaps
% if there is a needed pagebreak nearby.
\newcommand{\blankline}{\quad\pagebreak[2]}

%

%%%%%%%%%%%%%%%%%%%%%%%% End Helper Commands %%%%%%%%%%%%%%%%%%%%%%%%%%%
%%%%%%%%%%%%%%%%%%%%%%%%% Begin CV Document %%%%%%%%%%%%%%%%%%%%%%%%%%%%

\begin{document}
\makeheading{Timo J. Sand}

\section{Yhteystiedot}
%
% NOTE: Mind where the & separators and \\ breaks are in the following
%       table.
%
% ALSO: \rcollength is the width of the right column of the table
%       (adjust it to your liking; default is 1.85in).
%
\newlength{\rcollength}\setlength{\rcollength}{2.15in}%
%
\begin{tabular}[t]{@{}p{\textwidth-\rcollength}p{\rcollength}}
Timo Sand                           & \textit{Matkapuhelin:} (+49) 176 856 17539 \\
Klara-Franke-Straße 8            & \textit{E-mail:} \href{mailto:timo.sand@iki.fi}{timo.sand@iki.fi}\\
10577, Berlin
                                    & \textit{WWW:} \href{https://github.com/deiga/}{https://github.com/deiga/}\\
\end{tabular}

\section{Syntymäpäivä} 05.12.1985

\section{Ammatilliset kiinnostuksenkohteet}
%
Front-end Web Development, DevOps, Machine learning \& Neural networks

\blankline

\section{Koulutus}
%
\begin{outerlist}

\item[] B.Sc. [Oletettu],
        \href{http://cs.helsinki.fi/}
             {Helsingin yliopisto, Tietojenkäsittelytieteen laitos}, Kesä 2018
        \begin{innerlist}
        \item Opintoala: Koneoppiminen, Ohjelmistotuotanto
        \item Kansainvälinen vaihto: \href{http://bath.ac.uk/}
             {University of Bath, Department of Computer science}, Syyskuu 2012 - Kesäkuu 2013
        \end{innerlist}

\item[] Ylioppilastutkinto,
        \href{http://www.dsh.fi/Auf-Deutsch/Startseite}
             {Deutsche Schule Helsinki (Helsingin saksalainen koulu)}, Kesäkuu 2005

\end{outerlist}
\blankline

\section{Työkokemus}

\href{http://www.wunderdog.fi/}{\textbf{Wunderdog Oy}}
\begin{outerlist}

  \item[] \textit{Senior Software Developer \& Consultant}%
  \hfill \textbf{Syyskuu 2015 - }
  \begin{innerlist}
    \item Vastuullani on ison verrkoviihdepalvelun sivuston ylläpito ja kehittäminen\\
    \item Vastuullani oli luoda automaattinen provisiointi ison järjestelmän uusille instansseille. Ansiblella hallinoiddaan AWS servereitä ja asennetaan tarvittavat komponentit konfiguraatioineen.\\

    \emph{Käytetyt teknologiat}: JavaScript, Java, Ruby, Node.js, ReactJS, ES6, Ansible, HTML5, CSS3, AWS\\
  \end{innerlist}
\end{outerlist}
\blankline

\href{http://www.uusimuste.fi/}{\textbf{Kirjakosmos Oy}}
\begin{outerlist}

  \item[] \textit{Partner \& Software Developer}%
  \hfill \textbf{Helmikuu 2016 - }
  \begin{innerlist}
    \item Rakennamme Indiekirjakauppaa, jossa kuka tahans avoi julkaista kirjansa sähköisessä muodossa\\

    \emph{Käytetyt teknologiat}: JavaScript, Python, Node.js, AngularJS, ES6, Ansible, HTML5, CSS3, AWS, Django\\
  \end{innerlist}
\end{outerlist}
\blankline

\href{http://verkkokauppa.com/}{\textbf{Verkkokauppa.com Oyj}}
\begin{outerlist}

  \item[] \textit{Front-end Developer}%
  \hfill \textbf{Maaliskuu 2015 - Heinäkuu 2015}
  \begin{innerlist}
    \item Vastuullani oli kehittää uusia toiminnallisuuksia yrityksen verkkopalveluihin, joista esimerkkinä yrityksen verkkokauppa, kassa- ja ammattilaisjärjestelmä, sekä responsiivinen mobiilisivusto. Olin mukanan toteuttamassa verkkokaupan uutta ostoprosessia, sekä lisäämässä mobiilisivuston ominaisuuksia.\\

    \emph{Käytetyt teknologiat}: JavaScript, Node.js, ReactJS, AngularJS, ES6, Ansible, Ruby, HTML5, CSS3\\
  \end{innerlist}
\end{outerlist}
\blankline

\href{http://eficode.fi/}{\textbf{Eficode Oy}}
\begin{outerlist}

  \item[] \textit{Software Developer}%
  \hfill \textbf{Kesäkuu 2014 - Maaliskuu 2015}
  \begin{innerlist}
    \item Eficodella ollessani osallistuin useaan asiakasprojektiin seuraavissa tehtävissä: front-end kehittäjä ja full-stack kehittäjä. Muutamat kohokohdat näistä projekteista ovat Reposive, ePassi, sekä Spontaanipalaute.
Reposive oli JavaScriptillä toteutettu työpöytäsovellus, joka synkronoi Git-projektit pilvipalvelun kanssa; vastasin niin pilvipalvelun, kuin työpöytäsovelluksen jatkokehittämisestä.
ePassi on verkkopalvelu, joka tarjoaa kulttuuri-, urheilu- ja lounasedut sähköisesti; vastasin lounasedun maksamislogiikan uusimisesta ja toteututin usean markkinointisivun.
Spontaanipalaute on terveydenhuollon asiakaspalautejärjestelmä, johon suunnittelin arkkitehtuurin ja toteutin niin asiakaskäyttöliittymää, kuin ammattilaiskäyttöliittymää.\\
    \item Toimin front-end kehittäjänä projektissa, jossa tehtiin Knockout.js:n, Node.js:n ja Webkitin avulla työpöytäsovellusta Git-projektien pilvi synkronisaatiota varten.
      Toimin full-stack ohjelmistokehittäjänä projektissaja vastasin arkkitehtuurin suunnittelusta.\\

    \emph{Käytetyt teknologiat}: Javascript, Node.js, Knockout.js, Ruby, Ruby on Rails, Java, Spring 4, Spring Cloud, AngularJS, Docker, Ansible\\
    \emph{Taidot}: Konsultointi, tarjousten tekeminen, kommunikaatio
  \end{innerlist}
\end{outerlist}
\blankline


\href{http://steeri.fi/}{\textbf{Steeri Oy}}
\begin{outerlist}

\item[] \textit{Senior Software Designer}
    \hfill \textbf{Elokuu 2010 - Huhtikuu 2014}
    \begin{innerlist}
		\item Toimin Development \& Integration -tiimissä integraatio-asiantuntijana. Olen vastannut Steerin Continous Integration -järjestelmistä, versionhallinnan integroimisesta työnkulkuun ja tuotteiden julkaisujen hallinnasta. Tehtäviini on kuulunut sisäisen tuotteen kehittämistä, integraatioiden rakentamista eri järjestelmien välille sekä Unix palvelinten ylläpitoa.\\


         \emph{Käytetyt teknologiat}: Java, Spring, Hibernate, Ruby, JavaSript, Node.js, AngularJS, Apache HTTP, Robotframework\\
         \emph{Taidot}: Tiimityöskentely, kommunikaatio, paineenhallinta, kriittisiin muutoksiin reagoiminen, opettaminen
    \end{innerlist}
\end{outerlist}
\blankline

\href{http://www.helsinki.fi/university/index.html}{\textbf{University of Helsinki}}
\begin{outerlist}

\item[] \textit{Projektisihteeri}%
    \hfill \textbf{Marraskuu 2009 - Huhtikuu 2010}
    \begin{innerlist}
        \item Toimin TUHAT-projektissa teknisenä apuhenkilönä. Tehtäviini kuului suunnitella ja dokumentoida järjestelmän testausta, suorittaa järjestelmän kuormitustestaus, varmentaa järjestelmän tarjoaman verkkopalvelun toiminnan sekä monipuolisen tiedon kartoittamista ulkoisen palvelun avulla.\\

        \emph{Käytetyt teknologiat}: Python, JMeter\\
        \emph{Taidot}: Ulkopuolisen toimittajan kanssa kommunikointi, käytettävyysarviointi, Internet ohjelman kuormitustestaus
    \end{innerlist}
\end{outerlist}
\blankline

\href{http://roydon.fi}{\textbf{Roydon Ky}}
\begin{outerlist}

\item[] \textit{Web Developer}%
    \hfill \textbf{Syyskuu 2007 - Maaliskuu 2014}
    \begin{innerlist}
        \item Vastasin yrityksen nettisivujen suunnittelusta ja toteutuksesta. Vuosina 2010 — 2015 uusin sivut, sekä suunnittelin ja toteutin verkkokaupan yritykselle.\\

        \emph{Käytetyt teknologiat}: PHP, HTML, CSS, JavaScript, jQuery, Ruby on Rails\\
    \end{innerlist}
\end{outerlist}
\blankline

\clearpage


\section{Muu toiminta}
%
\begin{outerlist}

\item[] \textit{Järjestöt}
    \begin{innerlist}
        \item Perustajajäsen, Star Realms Finland, 2015-
        \item IT Officer, International Students Association, Students' Union, University of Bath, 2012-2013
        \item Webmaster, SATY ry (Suomen Airedalenterrieri etujärjestö), 2010-2011
        \item Webmaster, TKO-Äly ry (Tietojenkäsittelytieteen opiskelijajärjestö, Helsingin yliopisto), 2009-2010
    \end{innerlist}
\item[] \textit{Harrastukset}
    \begin{innerlist}
        \item Vapaa-ajallani edistän useaa open source -projektia. Esimerkkeinä:
          \begin{innerlist}
            \item \href{https://github.com/geerlingguy/ansible-role-elasticsearch}{Ansbile Role Elasticsearch}: Ansible rooli Elasticsearchin asentamiseen ja konfigurointiin.
            \item \href{https://github.com/deiga/Spotilist}{Spotilist}: Spotify-linkkien selvittämistä varten oleva verkkopalvelu.
            \item \href{https://github.com/donatorIO/donator}{donator}: UltraHack 2015 voittaja.
            \item \href{https://github.com/philc/vimium}{Vimium}: Chrome-laajennos, joka lisää VIM näppäinkomentoja selaimeen.
            \item \href{https://github.com/deiga/Carpenter}{Carpenter}: Verkkopalvelu, joka tarjoaa mahdollisuuden Steam pelien omistajuuden vertailuun kavereiden kesken.
          \end{innerlist}
        \item Hackathonit:
          \begin{innerlist}
            \item \href{http://2015.ultrahack.org/}{UltraHack 2015}: Rakensimme verkkosivun, jota kautta kättäjä pystyy merkitsemään itsensä johonkin paikkaan (kauppa, leffateatteri, kahvila, yms.) ja tämän paikan omistja lahjoittaa pienen summan hyväntekeväisyyteen. Creative Charity -osa-alueen voittaja \href{http://donator.io/}{donator.io}
            \item \href{https://www.eventbrite.de/e/botcamp-build-the-conversation-tickets-25003495127#}{Red Bull Bot Camp 2016}: Rakensimme Facebook Messenger Chat Botin, joka tarjoaa käyttäjälle Red Bull Urheilijoiden kuratoimaa sisältöä \href{http://https://github.com/BotcampBerlin/redbull-botcamp/}{Red Bull Wingbot}
          \end{innerlist}
        \item Luen fantasia- ja tieteiskirjallisuutta, sekä kirjoitan runoja ja tarinoita
        \item Frisbee Golf, Pyöräily, Boulderointi
        \item Tietokone-, Lauta- ja Roolipelit
    \end{innerlist}
\end{outerlist}

\section{Taidot}
%
\begin{outerlist}

\item[] \textit{Kielitaito}
    \begin{innerlist}
        \item Äidinkieli tai vastaava: Suomi, Saksa
        \item Ammatillinen pätevyys: Englanti
        \item Rajoitettu ammatillinen pätevyys: Ruotsi
    \end{innerlist}

\item[] \textit{Ohjelmointikielet}
    \begin{innerlist}
        \item Ruby, Python, Java, C++, C
        \item HTML5, CSS3, XML, JavaScript, SQL
        \item Haskell, Clojure, Google Go, Scala, Elm
    \end{innerlist}

\item[] \textit{Työkalut}
    \begin{innerlist}
        \item Ruby on Rails, Spring, Node.js, AngularJS, React, jQuery, Sinatra
        \item Versionhallinta: Git, Mercurial, Subversion, CVS
        \item Provisiointi: Ansible, Chef
        \item Testaus: JUnit, Robot Framework, Cucumber, Rspec
        \item Käyttöjärjestelmät: Mac OS X, Linux, Windows
        \item Server: Apache HTTP, nginx
        \item Scrum, Agile, Kanban, Behaviour-driven development, Test-driven development, Acceptance test-driven development
        \item Toimisto-ohjelmistot: Microsoft Office, OpenOffice
    \end{innerlist}

\end{outerlist}

\end{document}

%%%%%%%%%%%%%%%%%%%%%%%%%% End CV Document %%%%%%%%%%%%%%%%%%%%%%%%%%%%%
