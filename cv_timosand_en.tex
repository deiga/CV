%%%%%%%%%%%%%%%%%%%%%%%%%%%%%%%%%%%%%%%%%%%%%%%%%%%%%%%%%%%%%%%%%%%%%%%%
%%%%%%%%%%%%%%%%%%%%%% Simple LaTeX CV Template %%%%%%%%%%%%%%%%%%%%%%%%
%%%%%%%%%%%%%%%%%%%%%%%%%%%%%%%%%%%%%%%%%%%%%%%%%%%%%%%%%%%%%%%%%%%%%%%%

%%%%%%%%%%%%%%%%%%%%%%%%%%%%%%%%%%%%%%%%%%%%%%%%%%%%%%%%%%%%%%%%%%%%%%%%
%% NOTE: If you find that it says                                     %%
%%                                                                    %%
%%                           1 of ??                                  %%
%%                                                                    %%
%% at the bottom of your first page, this means that the AUX file     %%
%% was not available when you ran LaTeX on this source. Simply RERUN  %%
%% LaTeX to get the ``??'' replaced with the number of the last page  %%
%% of the document. The AUX file will be generated on the first run   %%
%% of LaTeX and used on the second run to fill in all of the          %%
%% references.                                                        %%
%%%%%%%%%%%%%%%%%%%%%%%%%%%%%%%%%%%%%%%%%%%%%%%%%%%%%%%%%%%%%%%%%%%%%%%%

%%%%%%%%%%%%%%%%%%%%%%%%%%%% Document Setup %%%%%%%%%%%%%%%%%%%%%%%%%%%%

% Don't like 10pt? Try 11pt or 12pt
\documentclass[10pt,english,a4paper]{article}
\usepackage{babel}
\usepackage[utf8]{inputenc}

% To allow comment environment
\usepackage{verbatim}


% This is a helpful package that puts math inside length specifications
\usepackage{calc}

% Simpler bibsection for CV sections
% (thanks to natbib for inspiration)
\makeatletter
\newlength{\bibhang}
\setlength{\bibhang}{1em}
\newlength{\bibsep}
 {\@listi \global\bibsep\itemsep \global\advance\bibsep by\parsep}
\newenvironment{bibsection}
    {\minipage[t]{\linewidth}\list{}{%
        \setlength{\leftmargin}{\bibhang}%
        \setlength{\itemindent}{-\leftmargin}%
        \setlength{\itemsep}{\bibsep}%
        \setlength{\parsep}{\z@}%
        }}
    {\endlist\endminipage}
\makeatother

% Layout: Puts the section titles on left side of page
\reversemarginpar

%
%         PAPER SIZE, PAGE NUMBER, AND DOCUMENT LAYOUT NOTES:
%
% The next \usepackage line changes the layout for CV style section
% headings as marginal notes. It also sets up the paper size as either
% letter or A4. By default, letter was used. If A4 paper is desired,
% comment out the letterpaper lines and uncomment the a4paper lines.
%
% As you can see, the margin widths and section title widths can be
% easily adjusted.
%
% ALSO: Notice that the includefoot option can be commented OUT in order
% to put the PAGE NUMBER *IN* the bottom margin. This will make the
% effective text area larger.
%
% IF YOU WISH TO REMOVE THE ``of LASTPAGE'' next to each page number,
% see the note about the +LP and -LP lines below. Comment out the +LP
% and uncomment the -LP.
%
% IF YOU WISH TO REMOVE PAGE NUMBERS, be sure that the includefoot line
% is uncommented and ALSO uncomment the \pagestyle{empty} a few lines
% below.
%

%% Use these lines for letter-sized paper
\usepackage[paper=letterpaper,
            %includefoot, % Uncomment to put page number above margin
            marginparwidth=1.2in,     % Length of section titles
            marginparsep=.05in,       % Space between titles and text
            margin=1in,               % 1 inch margins
            includemp]{geometry}

%% Use these lines for A4-sized paper
%\usepackage[paper=a4paper,
%            %includefoot, % Uncomment to put page number above margin
%            marginparwidth=30.5mm,    % Length of section titles
%            marginparsep=1.5mm,       % Space between titles and text
%            margin=25mm,              % 25mm margins
%            includemp]{geometry}

%% More layout: Get rid of indenting throughout entire document
\setlength{\parindent}{0in}

%% This gives us fun enumeration environments. compactitem will be nice.
\usepackage{paralist}

%% Reference the last page in the page number
%
% NOTE: comment the +LP line and uncomment the -LP line to have page
%       numbers without the ``of ##'' last page reference)
%
% NOTE: uncomment the \pagestyle{empty} line to get rid of all page
%       numbers (make sure includefoot is commented out above)
%
\usepackage{fancyhdr,lastpage}
\pagestyle{fancy}
%\pagestyle{empty}      % Uncomment this to get rid of page numbers
\fancyhf{}\renewcommand{\headrulewidth}{0pt}
\fancyfootoffset{\marginparsep+\marginparwidth}
\newlength{\footpageshift}
\setlength{\footpageshift}
          {0.5\textwidth+0.5\marginparsep+0.5\marginparwidth-2in}
\lfoot{\hspace{\footpageshift}%
       \parbox{4in}{\, \hfill %
                    \arabic{page} of \protect\pageref*{LastPage} % +LP
%                    \arabic{page}                               % -LP
                    \hfill \,}}

% Finally, give us PDF bookmarks
\usepackage{color,hyperref}
\definecolor{darkblue}{rgb}{0.0,0.0,0.3}
\hypersetup{colorlinks,breaklinks,
            linkcolor=darkblue,urlcolor=darkblue,
            anchorcolor=darkblue,citecolor=darkblue}

%%%%%%%%%%%%%%%%%%%%%%%% End Document Setup %%%%%%%%%%%%%%%%%%%%%%%%%%%%


%%%%%%%%%%%%%%%%%%%%%%%%%%% Helper Commands %%%%%%%%%%%%%%%%%%%%%%%%%%%%

% The title (name) with a horizontal rule under it
%
% Usage: \makeheading{name}
%
% Place at top of document. It should be the first thing.
\newcommand{\makeheading}[1]%
        {\hspace*{-\marginparsep minus \marginparwidth}%
         \begin{minipage}[t]{\textwidth+\marginparwidth+\marginparsep}%
                {\large \bfseries #1}\\[-0.15\baselineskip]%
                 \rule{\columnwidth}{1pt}%
         \end{minipage}}

% The section headings
%
% Usage: \section{section name}
%
% Follow this section IMMEDIATELY with the first line of the section
% text. Do not put whitespace in between. That is, do this:
%
%       \section{My Information}
%       Here is my information.
%
% and NOT this:
%
%       \section{My Information}
%
%       Here is my information.
%
% Otherwise the top of the section header will not line up with the top
% of the section. Of course, using a single comment character (%) on
% empty lines allows for the function of the first example with the
% readability of the second example.
\renewcommand{\section}[2]%
        {\pagebreak[2]\vspace{1.3\baselineskip}%
         \phantomsection\addcontentsline{toc}{section}{#1}%
         \hspace{0in}%
         \marginpar{
         \raggedright \scshape #1}#2}

% An itemize-style list with lots of space between items
\newenvironment{outerlist}[1][\enskip\textbullet]%
        {\begin{itemize}[#1]}{\end{itemize}%
         \vspace{-.6\baselineskip}}

% An environment IDENTICAL to outerlist that has better pre-list spacing
% when used as the first thing in a \section
\newenvironment{lonelist}[1][\enskip\textbullet]%
        {\vspace{-\baselineskip}\begin{list}{#1}{%
        \setlength{\partopsep}{0pt}%
        \setlength{\topsep}{0pt}}}
        {\end{list}\vspace{-.6\baselineskip}}

% An itemize-style list with little space between items
\newenvironment{innerlist}[1][\enskip\textbullet]%
        {\begin{compactitem}[#1]}{\end{compactitem}}

% To add some paragraph space between lines.
% This also tells LaTeX to preferably break a page on one of these gaps
% if there is a needed pagebreak nearby.
\newcommand{\blankline}{\quad\pagebreak[2]}

%

%%%%%%%%%%%%%%%%%%%%%%%% End Helper Commands %%%%%%%%%%%%%%%%%%%%%%%%%%%

%%%%%%%%%%%%%%%%%%%%%%%%% Begin CV Document %%%%%%%%%%%%%%%%%%%%%%%%%%%%

\begin{document}
\makeheading{Timo J. Sand}

\section{Contact Information}
%
% NOTE: Mind where the & separators and \\ breaks are in the following
%       table.
%
% ALSO: \rcollength is the width of the right column of the table
%       (adjust it to your liking; default is 1.85in).
%
\newlength{\rcollength}\setlength{\rcollength}{2.15in}%
%
\begin{tabular}[t]{@{}p{\textwidth-\rcollength}p{\rcollength}}
Timo Sand                           & \textit{Cell:} (+49) 176 856 17539 \\
Klara-Franke-Straße 8            & \textit{E-mail:} \href{mailto:hello@deiga.me}{hello@deiga.me}\\
10577, Berlin
                                    & \textit{WWW:} \href{https://github.com/deiga/}{https://github.com/deiga/}\\
\end{tabular}

\section{Date of Birth} 05.12.1985

\begin{comment}
\section{Professional Interests}
%
DevOps, Web App development, machine learning and neural networks
\end{comment}
\blankline

\section{Work Experience}

\href{http://www.wunderdog.fi/}{\textbf{Wunderdog GmbH}}
\begin{outerlist}

  \item[] \textit{Chief Technical Officer}%
  \hfill \textbf{March 2018 to current }
  \par
  I was responsible for recruiting developers, negotiating technical details of sales deals, acquiring new customers,
  staffing projects, planning growth trajectories of developers, evaluating the tech market and setting hiring requirements and building and evolving the company culture.
  During the time I was also hands-on in customer projects.
  \\
  \\
  \textbf{Projects}:
  \begin{description}
    \item [Smartly.io] Continued in the project.
    \item [Home Rental Platform] I was part of a team to figure out the current state of their infrastructure and what are the biggest pain points considering an upcoming cloud provider migration.
  \end{description}

  \emph{Non-technical skills}: Communication, Mentoring, Business Development, Leadership, Feedback culture, Technical sales, Creating growth trajectories, Hiring developers
  \par
  \emph{Used technologies}: Kotlin, Ruby, Python, Node.js, JavaScript, Groovy, AWS, Ansible, Docker, Kubernetes, Elm.js, Webpack, React, Jenkins\\
\end{outerlist}
\blankline

\href{http://www.wunderdog.fi/}{\textbf{Wunderdog Oy}}
\begin{outerlist}

  \item[] \textit{Senior Software Developer \& Consultant}
  \hfill \textbf{September 2015 to current }
  \par
  I was a fullstack consultant working on a number of customer projects. As a consultant, I was working as a part of the customer's team on-premises at the customer.

  When the Berlin office was founded I was promoted to CTO of the Berlin office.
  \\
  \\
  \textbf{Projects}:
  \begin{description}
    \item [Finnish ISP] I was part of the team developing the web portal for a Finnish VOD service.
    My role was a Fullstack developer with extensions into operations as I managed their CI environment.
    The backend was based on Java with a custom stack of libraries and the frontend was JavaScript based on jQuery, Bacon.js and Require.js which we started transitioning towards React.js

    \item [Recruitment Platform] I planned and built an Ansible setup to automate launching of new instances of the Platform for new customers into AWS.

    \item [Smartly.io] I was responsible for migrating an old and stale CI system to a modern one, which would be easier to keep up-to-date and would offload work to developer teams.
    The decision was made to work with Jenkins as a platform, but move as much as possible out of the hands of the CI system.
    So all projects were restructured to run tests inside containers and using Jenkins Pipelines all projects had the recipes for how the CI system should handle them.\par
    To all steps of the way, a mentality of configuration-as-code was applied which has allowed the team to create similar CI setups for other needs.
    A big challenge of the transition was Dockerizing previously highly interdependent projects.
    \par
    Additionally, I was heavily involved in automatic and building developer tooling, which allowed developers to reset their development environment to a stable status quo.
  \end{description}

  \emph{Non-technical skills}: Communication, Technology consulting, Mentoring, Presenting, Business development
  \par
  \emph{Used technologies}: JavaScript, Java, Ruby, Node.js, Shell, Kotlin, Python, Groovy, Bacon.js, React.js, Ansible, HTML5, CSS3, AWS, Webpack, Docker, Elm.js, Jenkins, Kubernetes\\
\end{outerlist}
\blankline

\href{http://www.uusimuste.fi/}{\textbf{Kirjakosmos Oy}}
\begin{outerlist}

  \item[] \textit{Co-Founder \& Software Developer}%
  \hfill \textbf{February 2016 to current}
  \par
  I joined the company just as they were ready to launch the initial version of Uusi Muste, an online indie book store.
  My responsibilities were to design and develop new features for the marketplace.\\

  \emph{Non-technical skills}: Listen to customer needs, working on own product
  \par
  \emph{Used technologies}: JavaScript, Python, Node.js, AngularJS, Ansible, HTML5, CSS3, AWS, Django, Clojure, Virtualbox, Serverless, PostgreSQL\\
\end{outerlist}
\blankline

\href{http://verkkokauppa.com/}{\textbf{Verkkokauppa.com Oyj}}
\begin{outerlist}
  \item[] \textit{Front-end Developer}%
  \hfill \textbf{March 2015 to July 2015}\par
    I was part of a team responsible for the development of new functionalities to the company’s online services, an example of these are the company’s e-commerce website, point of sale -system and the administration ERP, as well as responsive mobile site.
    I was involved in implementing a new e-commerce buying process, as well as to add features to the mobile site.
    \par
    \emph{Non-technical skills}: Product development workflow, building teams
    \par
    \emph{Used technologies}: PHP, JavaScript, Ruby, Node.js, Zend Framework, React.js, AngularJS, Ansible, HTML5, CSS3\\
\end{outerlist}
\blankline

\href{http://eficode.fi/}{\textbf{Eficode Oy}}
\begin{outerlist}
  \item[] \textit{Software Developer}%
  \hfill \textbf{June 2014 to March 2015}\par
    During my time at Eficode I took part in several customer projects as either a front-end or full-stack developer.
    \\
    \\
    \textbf{Projects}:
    \begin{description}
      \item [Healthcare Feedback System] I Planned and implemented micro-service architecture for the application.
      I implemented the client and administration faced interfaces.

      \item [Reposive] A desktop application that synchronizes your Git projects with a cloud service.
      I was responsible for developing the Prototype into an MVP. The desktop application was built with Webkit and Knockout.js and the Cloud service was built with Node.js

      \item [Digital Voucher System] In Finland employers compensate lunch and cultural and sports activities to employees.
      This is mostly handled by paper vouchers. I was part of developing a digital voucher system to replace the analogue one.
      I was responsible for renewing payment logic for lunch benefits and implementing multiple marketing pages.
    \end{description}

    \emph{Non-technical skills}: Technology consulting, making technical plans for offers, Communication
    \par
    \emph{Used technologies}: Javascript, Node.js, Knockout.js, Ruby, Ruby on Rails, Java, Spring Framework, Spring Cloud, AngularJS, Docker, Ansible, MySQL, HTML5, CSS3, MongoDB, Robot Framework\\
\end{outerlist}
\blankline

\href{http://steeri.fi/}{\textbf{Steeri Oy}}
\begin{outerlist}
  \item[] \textit{Senior Software Developer}%
  \hfill \textbf{August 2010 to April 2014}\par
    Working in Development \& Integration team as an integration specialist.
    I was responsible for the Continuos Integration systems, the Version control workflow
    and release management. My tasks ranged from developing our own product,
    building integration between customer systems, integrating customer systems
    to our own product, administration of unix servers.
    \par

    \emph{Non-technical skills}: Dealing with multiple stakeholders, working with remote teams, mentoring, communication, handling pressure, responding to critical changes
    \par
    \emph{Used technologies}: Java, Ruby, PHP, JavaScript, Node.js, Spring Framework, Primefaces, Hibernate, Apache HTTP, TYPO3, MySQL, MongoDB, MS SQL, Mule ESB, Jenkins, UNIX, nginx, Tomcat
\end{outerlist}
\blankline

\begin{comment} % COMMENTED OUT

\href{http://www.helsinki.fi/university/index.html}{\textbf{University of Helsinki}}
\begin{outerlist}

\item[] \textit{Project Assistant}%
    \hfill \textbf{November 2009 to April 2010}
    \begin{innerlist}
        \item Working as assistant for TUHAT project. Planning and documenting tests,
        creating and running load testing for the TUHAT system. Usability evaluation of the web service.
        Mapping and converting old data to new data.
        Tools used: Python, JMeter
        \emph{Skills}: Communication with foreign supplier, evaluation of usability
    \end{innerlist}
\end{outerlist}
\blankline


\href{http://corporate.elisa.fi/elisa-oyj/}{\textbf{Elisa Oyj}}
\begin{outerlist}

\item[] \textit{Customer advisor}%
    \hfill \textbf{June 2008 to November 2009}
    \begin{innerlist}
        \item Working as customer service representative in a contact center and private business exchange for over 100 businesses.
        \emph{Skills}: Quick adaptation to changing circumstances, handling pressure
    \end{innerlist}

\end{outerlist}
\blankline

\href{http://www.securitas.com/fi/fi/Services/Securitas-Tapahtumapalvelut/}{\textbf{Securitas Events Oy}}
\begin{outerlist}

\item[] \textit{Security Steward}%
    \hfill \textbf{October 2006 to April 2008}
    \begin{innerlist}
        \item Providing security in concerts, night-time restaurants, opening nights, nude baths
        \emph{Skills}: Communication, First-aid, Self-defense
    \end{innerlist}

\end{outerlist}
\blankline

\href{http://www.securitas.com/en/}{\textbf{Securitas Oy}}
\begin{outerlist}

\item[] \textit{Security Guard}%
    \hfill \textbf{September 2006 to January 2007}
    \begin{innerlist}
        \item Providing security against theft and violence in grocery stores
        \emph{Skills}: Communication, First-aid, Self-defense
    \end{innerlist}

\end{outerlist}
\blankline

\end{comment}

\section{Other activities}
%
\begin{outerlist}

\item[] \textit{Volunteer activities}
    \begin{innerlist}
        \item Mentor, NodeSchool Helsinki, 2016 to 2018
        \item Coach, RailsGirls Helsinki, April 2016
        \begin{comment}
        \item IT Officer, International Students Association, Students' Union, University of Bath, 2012 to 2013
        \item Webmaster, SATY ry (Finnish Airedaleterrier Association), 2010 to 2011
        \item Webmaster, TKO-Äly ry (Computer Science student organization, University of Helsinki), 2009 to 2010
        \end{comment}
    \end{innerlist}
\end{outerlist}

\section{Languages}
%
\begin{outerlist}

\item[]
    \begin{innerlist}
        \item Native or bilingual proficiency: Finnish, German
        \item Full professional proficiency: English
        \item Limited working proficiency: Swedish
    \end{innerlist}
\end{outerlist}
\begin{comment}
\section{Skills}
%
\begin{outerlist}

\item[] \textit{Languages}
    \begin{innerlist}
        \item Native or bilingual proficiency: Finnish, German
        \item Full professional proficiency: English
        \item Limited working proficiency: Swedish
    \end{innerlist}

\item[] \textit{Programming languages}
    \begin{innerlist}
        \item Ruby, Python, Java, C++, C
        \item HTML5, CSS, XML, JavaScript, SQL
        \item Haskell, Clojure, Google Go, Scala, Elixir
    \end{innerlist}

\item[] \textit{Tools}
    \begin{innerlist}
        \item Ruby on Rails, Spring, jQuery, Sinatra, NodeJS, SASS, LESS
        \item Database: PostgreSQL, MySQL, MS SQL, MongoDB
        \item Version control: Git, Mercurial, Subversion
        \item Operating System: Mac OS X, Linux, Windows
        \item Server: Apache HTTP
        \item Methods: Scrum, Agile, Kanban, Behaviour-driven development, Test-driven development
        \item Office software: Microsoft Office, OpenOffice
    \end{innerlist}
  \end{comment}

\end{outerlist}

\section{Education}
%
\begin{outerlist}

\item[] B.Sc. [Aborted],
        \href{http://cs.helsinki.fi/}
             {University of Helsinki, Department of Computer Science}, 2006 to 2016
        \begin{innerlist}
        \item Area of Study: Machine Learning, Software Engineering
        \item Exchange studies: \href{http://bath.ac.uk/}
             {University of Bath, Department of Computer Science}, September 2012 to June 2013
        \end{innerlist}
\begin{comment}
\item[] A-Levels,
        \href{http://www.dsh.fi/Auf-Deutsch/Startseite}
             {Deutsche Schule Helsinki (German school Helsinki)}, June 2005
\end{comment}
\end{outerlist}
\blankline

\end{document}

%%%%%%%%%%%%%%%%%%%%%%%%%% End CV Document %%%%%%%%%%%%%%%%%%%%%%%%%%%%%
